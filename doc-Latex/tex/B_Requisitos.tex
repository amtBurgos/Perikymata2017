\apendice{Especificación de Requisitos}

\section{Introducción}
En esta sección se hablará de los requisitos y casos de uso que han dado la hoja ruta a seguir para el desarrollo de la aplicación.
\section{Objetivos generales}
Los objetivos generales que persigue la aplicación, nos vienen dados desde el proyecto anterior \cite{perikymataV1}. Son los siguientes:
\begin{itemize}
    \item La aplicación debe poder unir las de imágenes de fragmentos del diente para ofrecer al usuario una imagen completa.
    \item Se debe poder aplicar un filtro o filtros para resaltar las perikymata de la imagen completa de la pieza dental. 
    \item Se debe poder delimitar la corona del diente en la imagen y dividirla en deciles sobre los que se dibujará una línea en la que se marcarán las perikymata detectadas.
    \item Todos los datos generados podrán ser guardados de forma persistente.
    \item Se deben corregir errores y mejorar cada una de las etapas mencionadas en la medida de lo posible, a fin de ser más eficaz en la detección de perikymata.
\end{itemize}

\pagebreak
\section{Catalogo de requisitos}
Los requisitos que debe satisfacer la aplicación son:
\begin{itemize}
        \item Se podrá introducir una imagen completa de una pieza dental. En caso de ser imágenes de fragmentos, se unirán formando la imagen completa.
        \item Se podrá preparar la imagen completa para su uso a lo largo de la aplicación.
        \item Se podrá aplicar un filtro a la imagen para resaltar las perikymata.
        \item Se detectarán y marcarán la perikymata en la imagen filtrada de manera automática, aunque el usuario podrá intervenir para corregir posibles errores.
        \item La aplicación guardará de manera automática las imágenes generadas y el usuario podrá exportar los datos generados.
        \item La aplicación deberá proporcionar las funcionalidades básicas de cualquier aplicación.
\end{itemize}

\section{Especificación de requisitos}

En este apartado se indicarán las especificación de los requisitos a cumplir.

\begin{itemize}
        \item RF-1: Introducir la imagen o imágenes de la pieza dental que usará la aplicación para conseguir una imagen completa.
            \begin{itemize}
                \item RF-1.1: Indicar un conjunto de imágenes y unirlas (\textit{stitching}).
                    \begin{itemize}
                        \item RF-1.1.1: Indicar un conjunto de imágenes válidas para unir que representan los fragmentos del diente.
                        \item RF-1.1.2: Unir la imágenes para obtener una imagen dental completa.
                    \end{itemize}
                \item RF-1.2: Indicar una imagen que represente la pieza dental completa.
            \end{itemize}              
                      
        \item RF-2: Preparar la imagen y datos para las etapas posteriores.
            \begin{itemize}
                \item RF-2.1: Rotar y recortar la imagen para orientar las perikymata de la forma más vertical posible y delimitar la zona de acción.
                \item RF-2.2: Indicar la escala a la que se encuentra la imagen de la pieza dental.      
            \end{itemize}
        
        \pagebreak        
        \item RF-3: Filtrar la imagen del diente y llevar a cabo la detección de perikymata.
            \begin{itemize}
                \item RF-3.1: Delimitar, de manera automática, los deciles de la corona dental.
                \item RF-3.2: Filtrar la imagen del diente gracias a un modo por defecto.
                \item RF-3.3: Filtrar la imagen del diente gracias a un modo avanzado con distintos parámetros de filtrado.
                \item RF-3.4: Dibujar una línea y marcar perikymata automáticamente.
                \begin{itemize}
                        \item RF-3.4.1: Dibujar una línea que atraviese las perikymata.
                        \item RF-3.4.2: Automarcar todas las perikymata que sean posibles. 
                \end{itemize}
                \item RF-3.5: Añadir o eliminar perikymata sobre la imagen filtrada.
            \end{itemize}        
        
        \item RF-4: Guardar todos los datos generados\footnote{Los cálculos realizados no se incluyen.}, incluidas las imágenes que intervienen en el proceso.
            \begin{itemize}
                \item RF-4.1: Guardar las imágenes que aporte el usuario y las que genere la aplicación en las carpetas del proyecto.
                \item RF-4.2: Almacenar, en formato \textit{csv}, los resultados generados sobre la distancia entre cada perikyma\footnote{\textit{Perikyma} es el singular de \textit{perikymata}.} y su localización en los deciles.
            \end{itemize}
            
        \item RF-5: Proporcionar acceso en todo momento a funcionalidades básicas de la aplicación.
            \begin{itemize}
                \item RF-5.1: Hacer uso de las \textit{migas de pan}.
                \begin{itemize}
                        \item RF-5.1.1: Ver la etapa actual de la aplicación en la que se encuentre el usuario.
                        \item RF-5.1.2: Retroceder a etapas anteriores de la aplicación.
                \end{itemize}
                \item RF-5.2: Abrir, cerrar y guardar el proyecto actual en cualquier momento.
                \item RF-5.3: Indicar la carpeta temporal que se usará en la operación de \textit{stitching}, sino, se usará la carpeta por defecto del sistema. 
                \item RF-5.4: Ayudar al usuario con explicaciones de cada componente por medio de \textit{tooltips}.
                \item RF-5.5: Permitir interactuar al usuario con terceras aplicaciones que utilice la propia aplicación.
            \end{itemize}
    
\end{itemize}    
    \pagebreak
    Como requisitos no funcionales encontraremos los siguientes:
    \begin{itemize}
            \item RNF-1: Se corregirán los errores reportados y los nuevos que surjan, en la medida de lo posible. 
            \item RNF-2: Se tratará de hacer más sencilla e intuitiva la aplicación.
            \item RNF-3: Se investigarán nuevas formas de implementación de cada etapa de la aplicación.
    \end{itemize}

\subsection{Diagrama de casos de uso}

En la figura \ref{fig:img/Requisitos/DiagramaCUPerikymataFINAL} podemos ver el diagrama general de casos de uso.

\imagenDos{img/Requisitos/DiagramaCUPerikymataFINAL}{Diagrama general de casos de uso}{1}

En las figuras \ref{fig:img/Requisitos/CU1}, \ref{fig:img/Requisitos/CU2}, \ref{fig:img/Requisitos/CU3}, \ref{fig:img/Requisitos/CU4}, \ref{fig:img/Requisitos/CU5} y \ref{fig:img/Requisitos/CU6} podemos observar la especificación de cada caso de uso.

\imagenDos{img/Requisitos/CU1}{Caso de uso 1}{1}\pagebreak
\imagenDos{img/Requisitos/CU2}{Caso de uso 2}{1}\pagebreak
\imagenDos{img/Requisitos/CU3}{Caso de uso 3}{1}\pagebreak
\imagenDos{img/Requisitos/CU4}{Caso de uso 4}{1}\pagebreak
\imagenDos{img/Requisitos/CU5}{Caso de uso 5}{1}\pagebreak
\imagenDos{img/Requisitos/CU6}{Caso de uso 6}{1}\pagebreak


