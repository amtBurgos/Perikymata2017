\capitulo{2}{Objetivos del proyecto}

\begin{comment}
Este apartado explica de forma precisa y concisa cuales son los objetivos que se persiguen con la realización del proyecto. Se puede distinguir entre los objetivos marcados por los requisitos del software a construir y los objetivos de carácter técnico que plantea a la hora de llevar a la práctica el proyecto.
\end{comment}

\section{Objetivos generales}

De la versión anterior del proyecto \cite{perikymataV1} heredamos una serie de requisitos generales acerca de la aplicación a desarrollar:
\begin{itemize}
    \item Se deben poder por unir los fragmentos para conseguir una imagen completa del diente.
    \item El usuario debe poder aplicar un filtro o filtros a fin de resaltar las perikymata. 
    \item La corona del diente se dividirá en diez partes (deciles) sobre la que el usuario dibujará una línea para poder automarcar las perikymata detectadas.
    \item Los datos conseguidos sobre el número de perikymata y su localización podrán ser exportados a un fichero.
\end{itemize}

Teniendo en cuenta esos objetivos, se profundizará en mejorar la detección y automarcado de las perikymata gracias a la aplicación de otros filtros de imagen y procedimientos más adecuados para resaltarlas.

\newpage

\section{Objetivos específicos}

\subsection{Unión de imágenes (\textit{stitching})}
Como mencionábamos en la introducción, la primera fase será la unión de imágenes (\textit{stitching} de aquí en adelante), que nos permitirá obtener una imagen completa de la pieza dental a partir de las imágenes de fragmentos tomadas por un microscopio electrónico. Los objetivos en esta fase serán:
\begin{itemize}
    \item Investigar, valorar y probar la utilización de otro lenguaje de programación para realizar el \textit{stitching}.
    \item Eliminar la dependencia total de archivos de bibliotecas de enlace dinámicos (\textit{dll}) que en la versión anterior necesitaban estar en la misma ruta que la aplicación Java para funcionar.
    \item Conseguir que la aplicación de \textit{stitching} sea multiplataforma de modo que no esté restringida a sistemas Windows de 64 bits. 
    \item Solucionar el error a la hora de hacer \textit{stitching} con imágenes en ciertas rutas.
\end{itemize}

\subsection{Filtrado de imagen y detección de perikymata}
El filtrado de la imagen completa del diente es la parte más importante del proyecto puesto que un buen filtrado permitirá resaltar mejor las perikymata. Los objetivos de esta fase son:
\begin{itemize}
    \item Investigar otros filtros de imagen relacionados con la detección de bordes.
    \item Realizar y documentar pruebas de los diferentes filtros, los distintos métodos empleados y como afectarán a la detección de líneas.
    \item Elaborar un procedimiento que, aplicado junto con los de nuevos filtros de imagen, permita marcar mejor las perikymata que con el filtro Prewitt \cite{wiki:Prewitt}.
    \item Emplear una técnica de detección de líneas adecuada en base al proceso de filtrado que se utilice finalmente.
\end{itemize}

\subsection{Aplicación}
A nivel de la aplicación marcaremos los siguiente objetivos:
\begin{itemize}
    \item Adaptar la aplicación Java para seleccionar el subprograma de \textit{stitching} adecuado en base al sistema operativo y la arquitectura en la que nos encontremos.
    \item Incluir la opción de seleccionar una carpeta para los archivos temporales del \textit{stitching}.
    \item Integrar adecuadamente en la aplicación el nuevo proceso de filtrado que pueda surgir.
    \item Incluir un modo sencillo de filtrado por defecto para facilitar al usuario la interacción con la aplicación y también, un modo avanzado con el que el usuario pueda modificar los posibles parámetros del filtrado.
\end{itemize}

\subsection{Corrección de errores}
\begin{itemize}
    \item Revisar elementos de la aplicación como el sistema de log y mejorar la eficacia en la fase de automarcado de las perikymata. 
    \item Localizar y solucionar el error que se produce en la carga de imágenes al reabrir un proyecto.
    \item Encontrar y solucionar los errores reportados por el cliente.
    \item Investigar otros fallos y, en caso de no conseguir solucionarlos, documentarlos como líneas de trabajo futuras.
\end{itemize}
